\documentclass{beamer}
\mode<presentation>
 \usetheme{CambridgeUS}
%\usepackage{bussproofs}
%\usetheme{Singapore}
%\usecolortheme{lily}
  %%\usetheme{Logictheme} %nothing else
  %\setbeamertemplate{footline}[frame number]
  
%%\usepackage{bussproofs}
%%\usepackage[latin1]{inputenc} 
%%\newcommand{\mypause}{}
%%\newcommand{\mypause}{\pause}


%\institute[Madrid]{Swansea Railway Verification Group, Critical Software Technologies, Invensys Rail\\ \quad}
%\title[Logic in Railway Verification]{Application of Logic to the Verification\\ of Railway Control Systems}
%\title[Surrey Workshop]{Modelling and Analyzing the European Rail Traffic Management %System (ERTMS)}
\title[Pythonic Parsing with Pyparsing]{Pythonic Parsing with Pyparsing}

\author[Andrew Lawrence]{Andrew Lawrence}
\date[PyCon UK, 15 September 2018]{PyCon UK 2018\\[1em]  Cardiff, 15 September 2018}
%\institute[Swansea University]{Joint work with Andrew Lawrence, Ulrich Berger,  Phil James, Markus Roggenbach}

\usepackage{tikz}
\usepackage{graphicx}
\usetikzlibrary{shapes}
\usetikzlibrary{positioning}
\usetikzlibrary{arrows}
\usepackage{tikz-uml}
\usepackage{verbatim}
\newcommand{\Red}{\mathbf{Red}}
\usepackage[nounderscore]{syntax}
\usepackage[linguistics]{forest}
%\usepackage{forest}

\newtheorem{mydef}{Definition}
\newtheorem{myremark}{Remark}
\newcommand{\ednote}[1]{{\bf #1}}
\newcommand{\fig}[1]{Fig.~\ref{fig:#1}}
\newcommand{\Val}{{\rm Val}}
\newcommand{\unprime}{{\rm unprime}}
\newcommand{\Vars}{{\rm Vars}}
\definecolor{bottomcol}{RGB}{222,222,222}


\begin{document}
\tikzstyle{class}=[
    rectangle,
    draw=black,
    text centered,
    anchor=north,
    text=black,
    text width=3cm,
    shading=axis,
    bottom color=bottomcol,top color=white,shading angle=45]


\begin{frame}
  \titlepage
\end{frame}


\begin{frame}

\frametitle{Workshop Overview}

\medskip
This workshop aims to give an introduction to parsing using the Pyparsing library
%\pause

\medskip

Overview:

%\begin{itemize}
%  \item \underline{Part I:} ERTMS -- what it is
% \item \underline{Part II:} ERTMS -- how it works
%  \item \underline{Part III:} Generic Modelling: ERTMS as a hybrid automaton
%  \item \underline{Part IV:} Encoding in Real-Time Maude   
%  \item \underline{Part V:} Verification \& simulation results
%\end{itemize}

\begin{itemize}
  \item \underline{Part I:} Parsing Introduction
  \item \underline{Part II:} First parser - 
  \item \underline{Part II:} Generic Modelling: ERTMS as a hybrid automaton
  \item \underline{Part I:} Encoding in Real-Time Maude   
  \item \underline{Part V:} Verification \& simulation results
\end{itemize}

\bigskip

\end{frame}

\section{Parsing Introduction}

%\begin{frame}
%\begin{center}
%{\Large ERTMS -- what it is}
%\end{center}

%\end{frame}


\tikzstyle{block} = [rectangle, draw, fill=blue!20, 
    text width=5em, text centered, rounded corners, minimum height=4em]
\tikzstyle{line} = [draw, -latex']
\tikzstyle{cloud} = [draw, ellipse,fill=red!20,
    minimum height=2em]
    

\begin{frame}
\frametitle{What is parsing?}
\begin{center}
\scalebox{0.8}
{
\begin{tikzpicture}[node distance = 2cm]
    % Place nodes
    \node [cloud] (init) {Source String};
    \node [block, below of=init] (lex) {Lexical Analysis};
    \node [cloud, below of=lex] (tokens) {Tokens};
    \node [block, below of=tokens] (syntax) {Syntactic Analysis};
    \node [cloud, below of=syntax] (parsetree) {Parse Tree};
    % Draw edges
    \path [line] (init) -- (lex);
    \path [line] (lex) -- (tokens);
    \path [line] (tokens) -- (syntax);
    \path [line] (syntax) -- (parsetree);
\end{tikzpicture}
}
\end{center}
\end{frame}

\begin{frame}
\frametitle{What is a grammar?}
\end{frame}


\begin{frame}[fragile]
\frametitle{Backus normal form (BNF)}
A Backus normal form (BNF) is metasyntax notation for describing grammars.

\begin{center}
  \begin{tabular}{ | c | c | }
    \hline
    element type & description \\ \hline\hline
    $<\mathrm{non terminals}>$ & non terminal symbols  \\ \hline
    \textbf{terminals} & terminal symbols  \\ \hline
    $|$ & choice  \\ \hline
    ::= & replaced-by  \\
    \hline
  \end{tabular}
\end{center}

\end{frame}

\begin{frame}[fragile]
\frametitle{Example Grammar}
\begin{grammar}
<sum> ::= <sum> $\mathbf{+}$ <product> | <product>

<product> ::= <product> $\mathbf{*}$ <value> | <value>

<value> ::= <int> | \textit{id}

<int> ::= <unsignedint> | $\mathbf{-}$<unsignedint>

<unsignedint> ::= <digit> | <unsignedint><digit>

<digit> ::= $\mathbf{0}$ | $\mathbf{1}$ | $\mathbf{2}$ | $\mathbf{3}$ | $\mathbf{4}$ | $\mathbf{5}$ | $\mathbf{6}$ | $\mathbf{7}$ | $\mathbf{8}$ | $\mathbf{9}$
 \end{grammar}
\end{frame}

\begin{frame}
\frametitle{Types of parsers}
\begin{center}
\begin{tikzpicture}[post/.style={->,shorten >=1pt,>=stealth',semithick}] 
\umlsimpleclass{Parser} 
\umlsimpleclass[x=-2, y=-3]{Top down} 
\umlsimpleclass[x=2, y=-3]{Bottom up}
\umlVHVinherit{Top down}{Parser} 
\umlVHVinherit{Bottom up}{Parser} 
\end{tikzpicture}
\end{center}
\end{frame}



\tikzset{
    invisible/.style={opacity=0,text opacity=0},
    visible on/.style={alt=#1{}{invisible}},
    alt/.code args={<#1>#2#3}{%
      \alt<#1>{\pgfkeysalso{#2}}{\pgfkeysalso{#3}} % \pgfkeysalso doesn't change the path
    },
}
\forestset{
  visible on/.style={
    for current and ancestors={
      /tikz/visible on={#1},
      edge={/tikz/visible on={#1}}}}}

\begin{frame}
\frametitle{Example Parse Tree}
\begin{center}
\scalebox{0.8} {
\begin{forest}
  for tree={
    if n children=0{
      font=\itshape,
      tier=terminal,
      l sep=20pt,
      minimum width=1.8cm
    }{},
  }
[sum
  [sum
    [product
      [product
        [value
          [id
            [$x$]
          ]
        ]
      ]
      [$*$]
      [value
        [int 
          [$2$]
        ]
      ]
    ]
  ]
  [$+$]
  [product 
    [value
      [int
        [$1$]
      ]
    ]
  ]
]
\end{forest}
}
\end{center}
\end{frame}

\begin{frame}
\frametitle{Bottom Up (LR) Parse}
\begin{center}
\scalebox{0.8} {
\begin{forest}
  for tree={
    if n children=0{
      font=\itshape,
      tier=terminal,
      l sep=20pt,
      minimum width=1.8cm
    }{},
  }
[sum (13), , visible on=<13->
  [sum (8), , visible on=<8->
    [product (7), visible on=<7->
      [product (3), visible on=<3->
        [value (2), visible on=<2->
          [id (1), visible on=<1->
            [$x$, visible on=<1->]
          ]
        ]
      ]
      [$\overset{(4)}{*}$, visible on=<4->]
      [value (6), visible on=<6->
        [int (5), visible on=<5->
          [$2$, visible on=<5->]
        ]
      ]
    ]
  ]
  [$\overset{(9)}{+}$, , visible on=<9->]
  [product (12) , visible on=<12->
    [value (11), visible on=<11->
      [int (10),  visible on=<10->
        [$1$,  visible on=<10->]
      ]
    ]
  ]
]
\end{forest}
}
\end{center}
\end{frame}
\tikzset{
    invisible/.style={opacity=0,text opacity=0},
    visible on/.style={alt=#1{}{invisible}},
    alt/.code args={<#1>#2#3}{%
      \alt<#1>{\pgfkeysalso{#2}}{\pgfkeysalso{#3}} % \pgfkeysalso doesn't change the path
    },
}
\forestset{
  visible on/.style={
    for tree={
      /tikz/visible on={#1},
      edge+={/tikz/visible on={#1}}}}}


\begin{frame}
\frametitle{Top Down Parse}
\begin{center}
\scalebox{0.8} {
\begin{forest}
  for tree={
    if n children=0{
      font=\itshape,
      tier=terminal,
      l sep=20pt,
      minimum width=1.8cm
    }{},
  }
[sum (1), visible on=<1->
  [sum (2), visible on=<2->
    [product (3), visible on=<3->
      [product (4), visible on=<4->
        [value (5), visible on=<5->
          [id (6), visible on=<6->
            [$x$, visible on=<6->]
          ]
        ]
      ]
      [$\overset{(7)}{*}$, visible on=<7->]
      [value (8), visible on=<8->
        [int (9), visible on=<9->
          [$2$, visible on=<9->]
        ]
      ]
    ]
  ]
  [$\overset{(10)}{+}$, visible on=<10->]
  [product (11), visible on=<11->
    [value (12), visible on=<12->
      [int (13), visible on=<13->
        [$1$, visible on=<13->]
      ]
    ]
  ]
]
\end{forest}
}
\end{center}
\end{frame}


\begin{frame}
\frametitle{More detailed description of recursive decent parsing}
\end{frame}

\begin{frame}
\frametitle{Why are are regular expressions bad you ask?}
\end{frame}

\begin{frame}
\frametitle{Pyparsing overview}
\end{frame}

\begin{frame}
\frametitle{Pyparsing Basics}
\end{frame}

\begin{frame}
\frametitle{Structuring a Simple Application}

\end{frame}

\begin{frame}
\frametitle{Words and literals}

\end{frame}

\begin{frame}
\frametitle{Combinators}
\end{frame}

\begin{frame}
\frametitle{Example 1: Floating point number}
\end{frame}

\begin{frame}
\frametitle{Exercise 1: Date time format string}

\end{frame}

\begin{frame}
\frametitle{Intermediate Pyparsing}

\end{frame}

\begin{frame}
\frametitle{Structuring parser results using Group, Combine and setName}
\end{frame}

\begin{frame}
\frametitle{Adding extra behaviour with setParseAction}
\end{frame}


\begin{frame}
\frametitle{Forward declaring parsers with Forward}
\end{frame}


\begin{frame}
\frametitle{Lookahead with NotAny and FollowedBy}
\end{frame}

\begin{frame}
\frametitle{Parsing dictionaries with Dict}
\end{frame}


\begin{frame}
\frametitle{Forward declaring parsers with Forward}
\end{frame}


\begin{frame}
\frametitle{Exercise 2: JSON Parser}
\end{frame}

\begin{frame}
\frametitle{Advanced Pyparsing}
\end{frame}


\begin{frame}
\frametitle{Debugging parsers}
\end{frame}

\begin{frame}
\frametitle{Whitespace Management}
\end{frame}


\begin{frame}
\frametitle{Packrat Parsing}
\end{frame}



\begin{frame}
\frametitle{Summary}
\end{frame}


\end{document}
\begin{frame}
\frametitle{References}    
 
%Specification Implementation Verification Method:

  \begin{thebibliography}{10}    
  \beamertemplatearticlebibitems
  \bibitem{} James, P., Lawrence, A.,  Moller, F., Roggenbach, M., Seisenberger, M. and Setzer, A.Chadwick, S. ,P. Kanso, K., :
\newblock{\em  Verification of solid state interlocking programs.}
\newblock In SEFM'13, LNCS 8368, Springer 2014.
\end{thebibliography} 

\bigskip
%Extraction of SAT and Resolution algorithm:

\begin{thebibliography}{10}    
  \beamertemplatearticlebibitems
  \bibitem{}
    Berger, U.,  Lawrence, A., Nordvall Forsberg, F. , Seisenberger, M. 
    \newblock {\em Extraction of Verified Decision Procedures}. LMCS, to appear. 

\bigskip

\bibitem{} Lawrence, A, Berger, U. , James, P., Roggenbach, M., Seisenberger, M.
\newblock {\em Modelling and Analysing the European Rail Traffic Control System}
\newblock FTSCS, 2014.

\end{thebibliography}
\end{frame}



\end{document}

\begin{frame}

\frametitle{European Rail Traffic Management System (ERTMS) II}

Formal methods and ERTMS:
\begin{itemize}

\item
EuRailCheck (2009): \\ suggested methodology and tools for
formalization and validation of the standard.

\item
Open ETCS Project (ongoing): \\ works towards an integrated framework for
modelling, development, implementation and testing of the standard.

\end{itemize}

Open research questions include:
\begin{enumerate}

\item
How can safety be verified? 

\item
How can capacity be measured and improved?

\item
How can reliability be measured and estimated?

\end{enumerate}

Here: 1 and, partially, 2.

\end{frame}

\section{Modelling and Analyzing ERTMS}

%\begin{frame}
%\begin{center}
%{\Large ERTMS -- how it works}
%\end{center}
%\end{frame}